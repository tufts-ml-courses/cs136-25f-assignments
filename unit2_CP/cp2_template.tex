\documentclass[10pt]{article}

%%% Doc layout
\usepackage{fullpage} 
\usepackage{booktabs}       % professional-quality tables
\usepackage{graphicx}       % figure graphics
\usepackage{microtype}      % microtypography
\usepackage{parskip}
\usepackage{times}


%% Hyperlinks always black, no weird boxes
\usepackage[hyphens]{url}
\usepackage[colorlinks=true,allcolors=black,pdfborder={0 0 0}]{hyperref}

%%% Math typesetting
\usepackage{amsmath,amssymb}

%%% Write out problem statements in purple, solutions in black
\usepackage{xcolor}
\newcommand{\officialdirections}[1]{{\color{purple} #1}}

%%% Avoid automatic section numbers (we'll provide our own)
\setcounter{secnumdepth}{0}


%% --------------
%% Header
%% --------------
\usepackage{fancyhdr}
\fancyhf{}
\setlength{\headheight}{15pt}
\fancyhead[C]{\ifnum\value{page}=1 Tufts CS 136 - 2025f - CP2 Submission \else \fi}
\fancyfoot[C]{\thepage} % page number
\renewcommand\headrulewidth{0pt}
\pagestyle{fancy}

%% --------------
%% Begin Document
%% --------------
\begin{document}

~\\ %% add vertical space
{\Large{\bf Student Name: TODO}}

~\\ %% add vertical space
{\bf Collaboration Statement:}

Total hours spent: TODO

I consulted the following human, textbook, or AI resources:
\begin{itemize}
\item TODO
\item TODO
\item $\ldots$	
\end{itemize}

By turning this document in, I attest that I have followed the 
\href{https://www.cs.tufts.edu/cs/136/2025f/index.html#collaboration}{[CS 136 Collaboration Policy]}. All solutions represent my own work. No solution text in this document was directly provided by other humans or artificial agents. 

\setcounter{tocdepth}{2}
\tableofcontents

\newpage

\subsection{1a}
\officialdirections{
Given a dataset of size $N$, how do we score the model's predictions? Translate the provided starter code for the score function of the MAP estimator into a mathematical expression involving our probabilistic model. Provide a function in terms of parameters $w_\text{MAP}, \beta$ and dataset $\{x_n,t_n\}_{n=1}^N$.}

TODO YOUR SOLUTION HERE

\subsection{1b}
\officialdirections{Fix and then execute \texttt{run\_MAP\_demo.py}. For each model order, report the $\beta$ that does best in terms of validation-set likelihood.}

\begin{table}[!h]
\begin{tabular}{r r r r}
model order & best $\beta$ for order & valid. score
\\
1 & TODO & TODO 
\\
4 & TODO & TODO 
\\
7 & TODO & TODO
\end{tabular}
\end{table}

\subsection{1c}
\officialdirections{Explain why different $\beta$ values might be preferred by different model orders. Hint: Inspect the visualizations produced from your 1b experiments.
}

TODO YOUR SOLUTION HERE

\subsection{1d}

\officialdirections{Compare the visuals produced by \texttt{run\_demo\_PPE.py} to those from the MAP estimator. What do you notice about the width of the 2-stddev intervals of the predictions at $x$ values far from the train data? Why does this suggest PPE might generalize better than MAP?}

TODO YOUR SOLUTION HERE
\newpage 

\subsection{2a: }
\renewcommand{\figurename}{Fig.}
\renewcommand{\thefigure}{2a}
 \begin{figure}[!h]
     \centering
     \includegraphics[width=0.95\textwidth]{example-image-a.pdf}
     \label{fig:2a}
\caption{TODO YOUR CAPTION HERE. 
}%endcaption
 \end{figure}

\newpage 

\subsection{2b}
\renewcommand{\thefigure}{2b}
 \begin{figure}[!h]
     \centering
     \includegraphics[width=0.6\textwidth]{example-image-b.pdf}
     \label{fig:2b}
\caption{
TODO YOUR CAPTION HERE
}%endcaption
 \end{figure}

\newpage 

\subsection{3a}
\renewcommand{\thefigure}{3a}
 \begin{figure}[!h]
     \centering
     \includegraphics[width=0.95\textwidth]{example-image-c.pdf}
     \label{fig:3a}
     \caption{
TODO YOUR CAPTION HERE
}%endcaption
 \end{figure}

\newpage 

\subsection{3b}
\renewcommand{\thefigure}{3b}
 \begin{figure}[!h]
     \centering
     \includegraphics[width=0.6\textwidth]{example-image-c.pdf}
     \label{fig:3b}
\caption{
TODO YOUR CAPTION HERE
}%endcaption
 \end{figure}
 
\end{document}
